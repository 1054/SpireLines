\documentclass[11pt,a4paper]{article}
\usepackage[margin=3cm]{geometry}
\usepackage[utf8]{inputenc}
\usepackage{courier}
\usepackage{amsmath}
\usepackage{amsfonts}
\usepackage{amssymb}
\usepackage{makeidx}
\usepackage{graphicx}
\usepackage{hyperref}
\usepackage{indentfirst}
\usepackage{xcolor}
\usepackage{epsfig}
\usepackage{listings}
% -- subtitle
\usepackage{titling}
\newcommand{\subtitle}[1]{%
	\posttitle{%
		\par\end{center}
	\begin{center}\large#1\end{center}
	\vskip0.5em}%
    % http://tex.stackexchange.com/questions/50182/subtitle-with-the-maketitle-page
}
% -- fonts
\usepackage{fontspec}
\setmainfont{RomanSerif}
\setsansfont{Overlock}
\setmonofont{Inconsolata}
% -- xmark
\newcommand{\xmark}{\textcolor{green!50!black}{x}}
% -- table font size -- http://tex.stackexchange.com/questions/27097/changing-the-font-size-in-a-table
\usepackage{floatrow}
\DeclareFloatFont{large}{\large} % "scriptsize" is defined by floatrow, "tiny" not
\floatsetup[table]{font=large}
% -- author, title
\author{}
\title{{\fontspec{Overlock SC}SpireLines} Herschel Photometry Manual}
%\subtitle{}
\begin{document}
\maketitle
\tableofcontents
\clearpage
\setlength{\baselineskip}{16pt}
\setlength{\parskip}{5pt}
% -- code style text box
\lstset{
	numbers=left,
	stepnumber=1,
	numbersep=10pt,
	numberstyle=\footnotesize,
	basicstyle=\fontsize{8}{12}\ttfamily,
	keywordstyle=\color{blue!50!black},
	stringstyle=\color{red!50!black},
	commentstyle=\color{red!50!green!50!blue!50},
	frame=shadowbox,
	rulesepcolor=\color{red!20!green!20!blue!20},
	escapeinside=``,
	xleftmargin=2em,
	xrightmargin=0em,
	aboveskip=1em,
	tabsize=4,
	showspaces=false,
	showstringspaces=false
}

%*************************************************************************************
\section{Abstract}

This is the manual for our Herschel far-infrared (FIR) photometry of nearby galaxies. We aim at accurately obtaining FIR luminosities of various physical scales, i.e. various Gaussian-shape beam areas, that are appropriate to the physical scales probed by the Herschel SPIRE FTS (Fourier Transform Spectrometer) molecular line data. 

FTS was an unique spectrometer which took the advantage of Fourier Transform and could observe a continuous wavelength range of $194 - 671 {\mu}m$ ($447 - 1545 \mathrm{GHz}$). Though the spectral resolution is only $\sim 1.44 \mathrm{GHz}$, its sensitivity still ensured the detections of rich molecular and atomic lines in more than 100 local galaxies, e.g. a full CO ladder $J=4\to3$ to $J=13\to12$ in many local luminous and ultra-luminous infrared galaxies (LIRGs and ULIRGs respectively; see the references of FTS CO in (U)LIRGs in Appendix \ref{Appendix_References_FTS_CO}). 

However, it should also be noticed that the size of FTS beam varies significantly across the wide wavelength range, e.g. from $\gtrsim43''$ to $\lesssim18''$, assuming Gaussian beam shape and taking the full width half maximum (FWHM). In order to study how the FTS spectral line properties are correlated with other gas/dust/metal/SF properties when the galaxy has a physical scale larger than the FTS beam, we should either scale FTS line properties to a common physical scale (e.g. the whole galaxy), or scale the other properties to each FTS line's physical scale. 

In this manual, we present both methods and results. 

TODO

We comprehensively correlate the obtained FIR luminosities with various gas/dust/metal/SF properties, e.g. molecular gas, dense molecular gas, dust temperature ($T_{\mathrm{dust}}$), the interstellar radiation field (ISRF) that dust grains are exposed to, gas phase metallicity (Z) and active galactic nuclear (AGN) activities. Finally, we investigate whether and how these properties are related to the star formation (SF) activities and intensities in galaxies. 

Therefore, in this manual, we will describe how we perform detailed photometry for nearby galaxies, and how we deal with various issues. 

TODO



\vspace{5cm}
Hints: black text are our method and procedures, \textcolor{blue}{blue text are notes}, and \textcolor{red}{red text are unsolved issues.}

%*************************************************************************************

\clearpage

%*************************************************************************************
\section{Flow Chart}
\label{FlowChart}

%\begin{figure}[H]
%	\caption{{Flow Chart of Processes.}\newline Descriptions.}
%	\centering
%	\includegraphics[width=1.05\textwidth]{FlowChartDiagram}
%\end{figure}

%*************************************************************************************

\clearpage

%*************************************************************************************
\section{Processing Status}
\label{ProcessingStatus}

%*************************************************************************************

\clearpage

%*************************************************************************************
\section{Herschel Photometry for FTS CO And H$_2$O Lines}

\subsection{FIR Light Profile}


\subsection{Convolving to Gaussian PSF with Aniano's Kernels}

%*************************************************************************************

\clearpage

%*************************************************************************************
\section{Herschel Photometry for Arbitrary Beam Sizes}

%*************************************************************************************

\clearpage

%*************************************************************************************
\section{Step 3}

%*************************************************************************************

\clearpage

%*************************************************************************************
\appendix

%*************************************************************************************
\section{Softwares}
\label{Appendix_Software_Dependencies}

\begin{lstlisting}[language=bash]
IDL
C++
\end{lstlisting}

%*************************************************************************************

\clearpage

%*************************************************************************************
\section{References on Herschel SPIRE FTS CO Data}
\label{Appendix_References_FTS_CO}

%*************************************************************************************

\clearpage

%*************************************************************************************













\end{document}